  \documentclass[a4paper,10pt]{report}
\usepackage[utf8]{inputenc}


\usepackage{amsmath}
\usepackage{amssymb}
\usepackage{amsthm}


%barrer hachurer 
\usepackage{ulem}

\newtheorem{thr}{Theorem} %[section]
\newtheorem{lm}{Lemma}[thr]
\newtheorem{slm}{Sub-lemma}[lm]
\newtheorem{sslm}{Sub-sub-lemma}[slm]
\newtheorem{pp}{Proposition}
\newtheorem{df}{Definition}
\newtheorem{rk}{Remark}

\newtheorem{td}{TODO}

\newcommand{\seg}[1]{[\![#1]\!]} 
\newcommand{\sego}[1]{\left [\![ #1 \right [\![} 
\newcommand{\segw}[1]{\seg{0..(\mid #1\mid -1)}} 
\newcommand{\ts}{\seg{0 .. (n-1)}}
\newcommand{\C}{\mathcal{C}_{A}}
\newcommand{\I}{\mathcal{I}_{A}}
\newcommand{\F}{\mathcal{F}_{A}}

\newcommand{\U}{\mathcal{U}_{A}}
\newcommand{\Xa}{\mathcal{X}}
\newcommand{\X}{\mathcal{X}_{n}}
\newcommand{\Ya}{\mathcal{Y}}
\newcommand{\Y}{\mathcal{Y}_{n}}
\newcommand{\pb}{$\mathcal{P}_{A}$}


\newcommand{\Lu}{$\mathcal{L}_{1}$}
\newcommand{\Ld}{$\mathcal{L}_{2}$}
\newcommand{\Lt}{$\mathcal{L}_{3}$}
\newcommand{\sm}{segm}
\newcommand{\sbe}{segbe}
\newcommand{\ses}{segs}
\newcommand{\sesp}{segs'}
\newcommand{\cupdot}{\mathbin{\mathaccent\cdot\cup}}

%tikz

\newcommand{\insertfigure}[3] {

\begin{minipage}[b]{#2\linewidth}
    \includegraphics[width=#2\linewidth]{#1} 
    \caption{#3} 
\end{minipage} }

\newcommand{\insertf}[2]{\insertfigure{#1}{2.0}{#2}}


%insertpicture in beamer
\newcommand{\insertpicture}[2]{
\begin{center}
  \begin{figure}
    \insertf{#1}{#2}
 \end{figure}
\end{center}
}


%tikz
\usepackage{tikz}
\usetikzlibrary{arrows,backgrounds,calc,trees,shapes,snakes,automata,backgrounds,petri,matrix}


\newcommand{\cmdtikz}
{
\tikzstyle{transition}=[circle,draw=black!75,minimum size=1cm]
  \tikzstyle{etou}=[transition,circle,fill=blue!75]
  \tikzstyle{check}= [transition,regular polygon,regular polygon sides=6,fill=red!75]
  \tikzstyle{store}= [transition,diamond,fill=green!75]
  \tikzstyle{move}= [transition,rectangle,fill=yellow!75]
  \tikzstyle{expl}= [transition,rectangle]
 }
\newcommand{\st}[2]{\node [store] (#1) [#2]{$Store$}}
\newcommand{\mv}[2]{\node [move] (#1) [#2]{$Move$}}
\newcommand{\expl}[3]{\node [expl] (#1) [#2]{#3}}

\newcommand{\et}[3]{\node [etou] (#1) [#2]{#3 $\wedge$}}
\newcommand{\ou}[3]{ \node [etou] (#1)  [#2] {#3 $\vee$}}

\newcommand{\eq}[3]{\node [check] (#1) [#2]{$eq_{#3}$}}
\newcommand{\peq}[3]{\node [check] (#1) [#2]{ $\overline{eq_{#3}}$}}
\newcommand{\tl}[3]{\node [check] (#1) [#2]{ $#3$}}
\newcommand{\ntl}[3]{\node [check] (#1) [#2]{ $\overline{#3}$}}
\newcommand{\rt}[1]{edge [pre] (#1)}
\newcommand{\etat}[2]{\node [transition,regular polygon,regular polygon sides=4] (q#1) [#2]  {$q_{#1}$}}

%insert tikz in report
\newcommand{\inserttikz}[2]{
\begin{center}
  \begin{figure}[h]
    \begin{tikzpicture}[node distance=2.5cm,>=stealth',bend angle=45,auto]
    \cmdtikz
    #1
\end{tikzpicture}



\caption{#2}
\end{figure}
\end{center}
}



% highlight edge :

\newcommand{\highe}[4]{\only<{#1}> {\draw[ultra thick,->, %Regular stuff
preaction={%But before that
draw,#2,-,% Draw yellow without any arrow head
double=#2,
double distance=1\pgflinewidth,
}] (#3) to (#4);
}
}

\newcommand{\hye}[3]{\highe{#1}{yellow}{#2}{#3}}

%highlight thread
\newcommand{\hyt}[2]{
\only<#1> {
\fill[yellow,opacity=0.5] \convexpath{#2.west, #2.east}{15pt};

}
}

%conex hull
\pgfdeclarelayer{background}
\pgfsetlayers{background,main}


\newcommand{\convexpath}[2]{
[   
    create hullnodes/.code={
        \global\edef\namelist{#1}
        \foreach [count=\counter] \nodename in \namelist {
            \global\edef\numberofnodes{\counter}
            \node at (\nodename) [draw=none,name=hullnode\counter] {};
        }
        \node at (hullnode\numberofnodes) [name=hullnode0,draw=none] {};
        \pgfmathtruncatemacro\lastnumber{\numberofnodes+1}
        \node at (hullnode1) [name=hullnode\lastnumber,draw=none] {};
    },
    create hullnodes
]
($(hullnode1)!#2!-90:(hullnode0)$)
\foreach [
    evaluate=\currentnode as \previousnode using \currentnode-1,
    evaluate=\currentnode as \nextnode using \currentnode+1
    ] \currentnode in {1,...,\numberofnodes} {
-- ($(hullnode\currentnode)!#2!-90:(hullnode\previousnode)$)
  let \p1 = ($(hullnode\currentnode)!#2!-90:(hullnode\previousnode) - (hullnode\currentnode)$),
    \n1 = {atan2(\x1,\y1)},
    \p2 = ($(hullnode\currentnode)!#2!90:(hullnode\nextnode) - (hullnode\currentnode)$),
    \n2 = {atan2(\x2,\y2)},
    \n{delta} = {-Mod(\n1-\n2,360)}
  in 
    {arc [start angle=\n1, delta angle=\n{delta}, radius=#2]}
}
-- cycle
}



%opening
\title{}
\author{}

\begin{document}

\maketitle

\begin{abstract}
\begin{td}
 I have to fill something sent by my teachers.
\end{td}

\end{abstract}


\section{Introduction.} 
I try to write this report to be concise and intuitive, the two proofs in the appendix are more formal.
I may repeat things cause the two proofs are written to be understood without the report.
\subsection{Problem considered.}
Usually, studying automatas and formal languages, we work with on a finite alphabet $\Sigma$.
But words over infinite alphabets are useful models in a variety of contexts in verification and databases.
That's why we can wonder what happens when we add to the alphabet $\Sigma$ an infinite set $D$.
An element of $D$ will be called data, and words on $\Sigma \times D$ will be called data words..
Precisely I studied automaton which are able to register a data and compare it to the current one.
I had to extend a result on those automaton on data word to a result on $n$-data words, which are words $\Sigma \times D^n$.
I begin to study three proofs of the emptiness decidability of a set of alternative register automaton on data words: 
\begin{td}
 quote 3 proof (1/2)
\end{td}
Then I could work on the emptiness decidability of those automaton on a subset of the $n$-data words (the set of ``nested words'').
To end with I worked on the emptiness decidability of my automaton on $n$-data words.
\subsection{Proceeding of the internship.} 
I begin to work in Cachan, I met Sylvain Smith three times to understand the three proofs, after I worked two months in Coventry becoming with Ranko Lazic and helped by Marcin Jurdzinski the two lasts weeks.
I was meeting them one time a week at the university of Warwick. With Ranko I wrote the two mains proofs of my internship, and with Marcin I worked on this report and the end of the writing of proofs.  

\section{My Definition of Data word.}
\begin{td}
  (1/2)
\end{td}
\section{Automaton studied.}
We will study alternative automaton which are able to register a $n$-tuple of data and to compare it (term to term) to the current one.
Formally we define:
\begin{df}[Nested Register Automata \textit{NARA}$_n$]
  A Nested Alternating Register Automaton (\textit{NARA}$_n$) is a tuple : 
  $\langle (\Sigma ,Q ,q_0, \delta )  \rangle$
  \begin{itemize}
   \item $\Sigma$ is an alphabet,
   \item $Q$ is a finite set of states,
   \item $q_0$ is the initial state ,
   \item $ \delta $ is a map from $Q $ to $\mathcal{T} $, where $\mathcal{T}$ is defined by the grammar : 
   $$ a \ | \  \overline{a} \ | \ Store(q) \ | \ eq_k \ | \  \overline{eq_k} \ | \  q \wedge q' \ | \ q \vee q' \ | \ Move(q)  $$
  where $a \in \Sigma$, $k \in  \ts$ and $q,q' \in Q$.
  \end{itemize}
\end{df}

Intuitively we will see that :
\begin{itemize}
   \item $ a \ | \  \overline{a}$ compare the current letter to $a$.
   \item $eq_k \ | \  \overline{eq_k}$ compare the current data to the registered one (just the $k^{th}$ term).
   \item $Store(q)$ register the current data.
   \item $Move(q)$ move on the word.
   \item $q \vee q' $ allow a choice between two states.
   \item $q \wedge q'$  duplicate a thread in two different one. 
\end{itemize}


A run is a succession of configuration that we define here :


\begin{df}[Configurations of an \textit{NARA}$_n$ and threads]
For $A$ an \textit{NARA}$_n$ we call configuration on $A$ a tuple  
 $\langle  a ,x_c, \Delta    \rangle$
 such that 
  \begin{itemize}
    \item $a \in \Sigma$ is the current letter, 
    \item $x_c\in  \X$ is the current data, 
    \item $\Delta \in \mathcal{P}_f(\X \times Q)$ is a finite set of threads, where a thread is a pair  
    $\langle q,x  \rangle$ such that:
      \begin{itemize}
	\item $q \in Q$ is the current state of the thread,
	\item $x\in  \X $ is the data in the register of the thread.
      \end{itemize}
   \end{itemize}
 We write $\C =  \Sigma \times \X \times \mathcal{P}_f(\X \times Q)$, the set of all configurations.
\end{df}

\begin{td} (1,5)
  exemples
\end{td}

\section{proof decidability} 

\begin{td} (2)
  PROOF
\end{td}

\section{Proof of undecidability}
\subsection{PCP To \Lt.}

\begin{thr}[Problem]
    The emptiness problem for \textit{NARA} on $n$-data is undecidable.
\end{thr}
The main idea of the proof is to reduce the following statement of PCP. 
 \begin{df}[PCP] A PCP problem can be defined like that :
    \begin{itemize}
     \item[Input:] $\Gamma$ be a finite alphabet $m \in \mathbb N$ be an integer and $u,v \in {(\Gamma^*)}^m$ be two tuples of pieces.
     \item[Question:] Does there exist $q \in \mathbb N$ and $h \in \sm ^ q$ such that $u_{h_1}...u_{h_q}  = v_{h_1}...v_{h_q}$.
    \end{itemize}
  \end{df}
  Then we set an input of PCP and we show that there is a solution if and only if the following language \Lu\ is empty.
  \begin{df}[\Lu] 
    \Lu $= \{w\#w'\#   \mid    
     q \in \mathbb N 
    \mid h \in \sm ^ q \mid \\\
     w = h_1u_{h_1}...h_qu_{h_q}  
    \wedge w' = h_1v_{h_1}...h_qv_{h_q}
    \wedge u_{h_1}...u_{h_q}  = v_{h_1}...v_{h_q}
    \}$.	
  \end{df}
  Hence, to show that our problem is undecidable, it remains to show that we can recognize \Lu.
  We will see that we are able to recognize \Lu\ cause we are able to recognize the language of double words $ww$.
  Precisely we will be able to recognize the following language \Lt:
  \begin{df}[\Lt($\Sigma_1,\Sigma_2$)]
    Let $\Sigma_1,\Sigma_2$ be two alphabets, 
    \Lt($\Sigma_1,\Sigma_2$) $= \{w\#w'\# \in (\Sigma_1 \cupdot \Sigma_2)^* \mid \Pi_{\Sigma_1}(w) = \Pi_{\Sigma_1}(w')\}$.
    Where $\Pi_{\Sigma_1}(w)$ is the projection on $\Sigma_1$ of the word w (ignoring letters of $\Sigma_2$).
  \end{df}.
  Indeed when we have shown that \Lt\ is recognizable we have made the harder cause : \Lu\ $=$ \Ld\ $\cap$ \Lt($\Gamma,\sm$) $\cap$ \Lt($\sm,\Gamma$).
  Where \Ld\ is the following regular language : 
  \begin{df}[\Ld]
    We define the language \Ld $= \left(\bigcup\limits_{i\in \sm} iu_i\right)^* \# \left(\bigcup\limits_{i\in \sm} iv_i\right)^* \# $
  \end{df}
  \subsection{To recognize \Lt.}
  We will construct an automaton to recognize \Lt.
    Intuitively to recognize $w\#w'\#$ with $ \Pi_{\Sigma_1}(w) =\Pi_{\Sigma_1}(w')$, we will open, for all letter of $\Sigma_1$ in $w$, a thread which check that: 
  \begin{itemize}
   \item this data won't appear again in $w$ 
   \item this data appears exactly one time in $w'$
   \item when it appears, it is on the same letter of $\Sigma_1$.
  \end{itemize}
  The main idea here is to use data to represent successive integers with $\sim_0$ and $\sim_1$ as we can see in the following picture, $\sim_0$ (respectively $\sim_1$) represents the successor on an even (respectively odd) number:
  \begin{td}(1/2)
    Picture
  \end{td} 
  This automaton is doing what we need 
  \begin{td}
    explain bloc by bloc
  \end{td}
    \inserttikz{
  \newcommand{\mon}[2]{\node [move] (#1) [#2]{$On \Sigma_1$}}
\newcommand{\egal}[2]{\node [check] (#1) [#2]{$\sim$}}
\newcommand{\fake}[2]{\node [dashed, transition , rectangle ] (#1) [#2]{}}

\node [etou] (first) {$\wedge$};
% \expl {oe1} {below right  of = first}{odd or even }
% \rt {first};
 \mon{fmon}{left of = first }
 edge [post] (first);
% 
 \node (begin) [draw = none, left of = fmon] {}
 edge [post] (fmon);


 
\expl {rmb} {above of = first}{duplicate($k$)}
edge [pre] node[left] {($k:=1$)} (first)
edge [dashed,loop above] node[above]{($k:= 1-k$)} (rmb) ;


\expl {dl} {right  of = rmb}{deal}
edge [pre] node[above] {(k)} (rmb) ;

\expl {suc1} { right  of = dl}{succesor($k$)}
edge [pre] node[above] {(a)} (dl) ;







\expl {nv1} {right of = suc1}{never}
 \rt {suc1};
% \et {et} {right of =nv1} {}
% \rt {nv1};
% 


\expl {fnd} { right  of = nv1}{found$(k)$}
\rt {nv1};


\expl{eg2} {below of = fnd} {same($a$)}
\rt {fnd};

\expl {suc1} {below of = eg2}{succesor($k$)}
\rt {eg2};


\expl {nv2} {left of = suc1}{never}
\rt {suc1};

\tl {l2} {left of = nv2} {\#}
\rt {nv2};


% 
% \expl {sb} {below  of = et}{check consistency}
% \rt {et}
% edge [post] (l2);



\expl{t0}{right of= first}{deal}
edge [pre] node[sloped,above] {($k:=0$)} (first);

\expl{et2}{right of= t0}{succesor(0)}
edge [pre] node[above] {(a)} (t0);

\expl{nv0}{right of= et2}{never}
\rt {et2}
edge [post]  (eg2) ;





  }{Automaton recognizing \Lt.}
  \newpage
  
  
 
  
\section{conclusion} 

  \begin{td}
    conclude
  \end{td}

\end{document}
