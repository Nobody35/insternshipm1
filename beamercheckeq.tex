\documentclass{beamer}
\usetheme{Singapore}
\usepackage{default}
\usepackage[utf8]{inputenc}
\usepackage[version=0.96]{pgf}


%tikz
\usepackage{tikz}
\usetikzlibrary{arrows,backgrounds,calc,trees,shapes,snakes,automata,backgrounds,petri}


\newcommand{\cmdtikz}
{
\tikzstyle{transition}=[circle,draw=black!75,minimum size=1cm]
  \tikzstyle{etou}=[transition,circle,fill=blue!75]
  \tikzstyle{check}= [transition,regular polygon,regular polygon sides=3,fill=red!75]
  \tikzstyle{store}= [transition,diamond,fill=green!75]
  \tikzstyle{move}= [transition,rectangle,fill=yellow!75]
  \tikzstyle{expl}= [transition,rectangle]
 }
\newcommand{\st}[2]{\node [store] (#1) [#2]{$Store$}}
\newcommand{\mv}[2]{\node [move] (#1) [#2]{$Move$}}
\newcommand{\expl}[3]{\node [expl] (#1) [#2]{#3}}

\newcommand{\et}[3]{\node [etou] (#1) [#2]{#3 $\wedge$}}
\newcommand{\ou}[3]{ \node [etou] (#1)  [#2] {#3 $\vee$}}

\newcommand{\eq}[3]{\node [check] (#1) [#2]{$eq_{#3}$}}
\newcommand{\peq}[3]{\node [check] (#1) [#2]{ $\overline{eq_{#3}}$}}
\newcommand{\tl}[3]{\node [check] (#1) [#2]{ $#3$}}
\newcommand{\ntl}[3]{\node [check] (#1) [#2]{ $\overline{#3}$}}
\newcommand{\rt}[1]{edge [pre] (#1)}

%insert tikz in beamer
\newcommand{\inserttikz}[2]{
\begin{center}
  \begin{figure}
    \scalebox{0.6}{
    \begin{tikzpicture}[node distance=2cm,>=stealth',bend angle=45,auto]
    \cmdtikz
    #1
\end{tikzpicture}

}

\caption{#2}
\end{figure}
\end{center}
}


%conex hull
\pgfdeclarelayer{background}
\pgfsetlayers{background,main}


\newcommand{\convexpath}[2]{
[   
    create hullnodes/.code={
        \global\edef\namelist{#1}
        \foreach [count=\counter] \nodename in \namelist {
            \global\edef\numberofnodes{\counter}
            \node at (\nodename) [draw=none,name=hullnode\counter] {};
        }
        \node at (hullnode\numberofnodes) [name=hullnode0,draw=none] {};
        \pgfmathtruncatemacro\lastnumber{\numberofnodes+1}
        \node at (hullnode1) [name=hullnode\lastnumber,draw=none] {};
    },
    create hullnodes
]
($(hullnode1)!#2!-90:(hullnode0)$)
\foreach [
    evaluate=\currentnode as \previousnode using \currentnode-1,
    evaluate=\currentnode as \nextnode using \currentnode+1
    ] \currentnode in {1,...,\numberofnodes} {
-- ($(hullnode\currentnode)!#2!-90:(hullnode\previousnode)$)
  let \p1 = ($(hullnode\currentnode)!#2!-90:(hullnode\previousnode) - (hullnode\currentnode)$),
    \n1 = {atan2(\x1,\y1)},
    \p2 = ($(hullnode\currentnode)!#2!90:(hullnode\nextnode) - (hullnode\currentnode)$),
    \n2 = {atan2(\x2,\y2)},
    \n{delta} = {-Mod(\n1-\n2,360)}
  in 
    {arc [start angle=\n1, delta angle=\n{delta}, radius=#2]}
}
-- cycle
}




\begin{document}

\begin{frame}{titre}
\inserttikz{

\tikzstyle{triangleinvi}= [transition,fill=white, regular polygon,regular polygon sides=3,dashed]
 \node [transition,regular polygon, regular polygon sides = 4] (q0)  {$q0$};
 \mv {move}{right of= q0} 
    edge [pre] (q0);
  \st {store}{right of= move}
    \rt {move};    
 \et {big}{right of= store}{}
    \rt{store}
    edge [post,bend right] (move);
 \ou {ou}{right of= big}{$k$ :}
    \rt{big};
 \node (invi6) [circle,dashed,above of=ou] {}
  edge [dashed,pre] (big);
 \node (invi7) [draw=none,below of=ou] {}
  edge [dashed,pre] (big);
 \ou {ou1}{above of= invi6}{$1$ :}
    \rt{big}
    edge [dashed] (ou);
 \node (invi4) [triangleinvi,below right of=ou1] {}
    edge [pre,dashed] (ou1);
 \node (invi5) [triangleinvi,right of=ou1] {}
  edge [pre,dashed] (ou1);
 \ou {oun}{below of= invi7}{$n$ :}
    \rt{big}
    edge [dashed] (ou);
 \node (invi2) [triangleinvi,right of=oun] {}
    edge [pre,dashed] (oun);
 \node (invi3) [triangleinvi,above right of=oun] {}
  edge [pre,dashed] (oun);
 
 \node (invi) [draw=none,right of=ou] {};
 \eq {eq}{above right of=invi}{k+1}
    \rt{ou};
 \peq {peq}{below right of=invi}{k}
    \rt{ou};
 
 
  
  

   \begin{pgfonlayer}{background}
  \pause
 %create
  \expl{cr}{below of=store}{create};
  \fill[green,opacity=0.3] \convexpath{move,store.north,store,cr,store,store.north,big}{23pt};
  
  \pause
  %delete
  \expl{dl1}{right of=eq}{delete};
  \fill[black,opacity=0.3] \convexpath{ou,eq,eq.north,eq,dl1}{23pt};
  
  \pause
  %delete
  \expl{dl2}{right of=peq}{delete};
  \fill[black,opacity=0.3] \convexpath{ou,peq,dl2,peq}{23pt};
  
  \pause
  %for 
  \node (invi2) [draw=none,left of=ou1] {};
   \expl{for}{left of=invi2} {for $k$ from $1$ to $n$};
  \fill[blue,opacity=0.3] \convexpath{for.west,for,ou1,big,ou1,ou,oun,big,ou,ou1}{23pt};
  
  
  \end{pgfonlayer}
 
}
{Progressively finer.}

\end{frame}



\end{document}

