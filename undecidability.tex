  \documentclass[a4paper,10pt]{report}
\usepackage[utf8]{inputenc}

\usepackage{amsmath}
\usepackage{amssymb}
\usepackage{amsthm}


%barrer hachurer 
\usepackage{ulem}


\newtheorem{thr}{Theorem} %[section]
\newtheorem{lm}{Lemma}[thr]
\newtheorem{slm}{Sub-lemma}[lm]
\newtheorem{sslm}{Sub-sub-lemma}[slm]
\newtheorem{pp}{Proposition}
\newtheorem{df}{Definition}
\newtheorem{rk}{Remark}

\newtheorem{td}{TODO}

\newcommand{\Pbb}{$\mathcal{P}$}

\newcommand{\Lu}{$\mathcal{L}_{1}$}
\newcommand{\Ld}{$\mathcal{L}_{2}$}
\newcommand{\Lt}{$\mathcal{L}_{3}$}
%tikz

\newcommand{\insertfigure}[3] {

\begin{minipage}[b]{#2\linewidth}
    \includegraphics[width=#2\linewidth]{#1} 
    \caption{#3} 
\end{minipage} }

\newcommand{\insertf}[2]{\insertfigure{#1}{2.0}{#2}}


%insertpicture in beamer
\newcommand{\insertpicture}[2]{
\begin{center}
  \begin{figure}
    \insertf{#1}{#2}
 \end{figure}
\end{center}
}


%tikz
\usepackage{tikz}
\usetikzlibrary{arrows,backgrounds,calc,trees,shapes,snakes,automata,backgrounds,petri,matrix}


\newcommand{\cmdtikz}
{
\tikzstyle{transition}=[circle,draw=black!75,minimum size=1cm]
  \tikzstyle{etou}=[transition,circle,fill=blue!75]
  \tikzstyle{check}= [transition,regular polygon,regular polygon sides=6,fill=red!75]
  \tikzstyle{store}= [transition,diamond,fill=green!75]
  \tikzstyle{move}= [transition,rectangle,fill=yellow!75]
  \tikzstyle{expl}= [transition,rectangle]
 }
\newcommand{\st}[2]{\node [store] (#1) [#2]{$Store$}}
\newcommand{\mv}[2]{\node [move] (#1) [#2]{$Move$}}
\newcommand{\expl}[3]{\node [expl] (#1) [#2]{#3}}

\newcommand{\et}[3]{\node [etou] (#1) [#2]{#3 $\wedge$}}
\newcommand{\ou}[3]{ \node [etou] (#1)  [#2] {#3 $\vee$}}

\newcommand{\eq}[3]{\node [check] (#1) [#2]{$eq_{#3}$}}
\newcommand{\peq}[3]{\node [check] (#1) [#2]{ $\overline{eq_{#3}}$}}
\newcommand{\tl}[3]{\node [check] (#1) [#2]{ $#3$}}
\newcommand{\ntl}[3]{\node [check] (#1) [#2]{ $\overline{#3}$}}
\newcommand{\rt}[1]{edge [pre] (#1)}
\newcommand{\etat}[2]{\node [transition,regular polygon,regular polygon sides=4] (q#1) [#2]  {$q_{#1}$}}

%insert tikz in report
\newcommand{\inserttikz}[2]{
\begin{center}
  \begin{figure}[]
    \begin{tikzpicture}[node distance=2.5cm,>=stealth',bend angle=45,auto]
    \cmdtikz
    #1
\end{tikzpicture}



\caption{#2}
\end{figure}
\end{center}
}
\newcommand{\inserttikzh}[2]{
\begin{center}
  \begin{figure}[!h]
    \begin{tikzpicture}[node distance=2.5cm,>=stealth',bend angle=45,auto]
    \cmdtikz
    #1
\end{tikzpicture}



\caption{#2}
\end{figure}
\end{center}
}



%todo check segment
\newcommand{\sm}{\{1..m\}}
\newcommand{\stt}{\{1..t\}}
\newcommand{\sbe}{\{b..e\}}
\newcommand{\ses}{\{0..s\}}
\newcommand{\sesp}{\{s+1..s'+s+1\}}
\newcommand{\cupdot}{\mathbin{\mathaccent\cdot\cup}}

\begin{document}
  We will now show that  this problem is undecidable:
  \begin{thr}
    The emptiness problem for \textit{NARA} on $n$-data is undecidable.
  \end{thr}
  \begin{rk}[Intuition of the proof.]
  The main idea of the proof is to reduce a statement of PCP. 
  Then we set an input of PCP and we show that there is a solution if and only if a certain language \Lu\ is empty.
  Hence, to show that our problem is undecidable, it remains to show that we can recognize \Lu.
  We will see that we are able to recognize \Lu\ cause we are able to recognize an other language \Lt\ near from the language of square words ($ww$).
  Indeed when we have shown that \Lt\ is recognizable we have made the harder cause we have a regular language \Ld\ such that  \Lu\ $=$ \Ld\ $\cap$ \Lt($\Gamma,\sm$) $\cap$ \Lt($\sm
  ,\Gamma$). 
  By the way, to recognize \Lt\ we will try to represent a counter with $\sim_0$ and $\sim_1$ (see figure).   
  \end{rk}
  Formaly we now assume, by absurde,  that it is decidable, and we will show then PCP is. 
  Let us define PCP.
  \begin{df}[PCP] PCP can be defined like that :
    \begin{itemize}
     \item[Input:] $\Gamma$ be a finite alphabet $m \in \mathbb N$ be an integer and $u,v \in {(\Gamma^*)}^m$ be two tupples of pieces.
     \item[Question:] Does there exist $q \in \mathbb N$ and $h \in {\sm} ^ q$ such that $u_{h_1} \ldots u_{h_q}  = v_{h_1}\ldots v_{h_q}$.
    \end{itemize}
  \end{df}
    Then, now, we set  $\Gamma$ be a finite alphabet $m \in \mathbb N$ be an integer and  $u,v \in {(\Gamma^*)}^m$ be two tupples of piece.
    And we will try to resolve this PCP statement.
    To improve the readability we will write $P = \sm$. 
  \begin{df}[\Lu] 
    We define the language \Lu $= \{w\#w'\# \in (P \cupdot \Gamma \cupdot \{\# \})^*  \mid \\\    
     q \in \mathbb N
    \mid h \in P ^ q
    \wedge w = h_1u_{h_1}\ldots h_qu_{h_q}  
    \wedge w' = h_1v_{h_1}\ldots h_qv_{h_q}
    \wedge u_{h_1}\ldots u_{h_q}  = v_{h_1}\ldots v_{h_q}
    \}$
  \end{df}
  We can first notice that :
  \begin{lm}
      If we can recognize \Lu\ then we can solve our PCP statement. 
  \end{lm}
  \begin{proof}
   If we can recognize \Lu, as we assumed can decide if \Lu\ is empty.
   By construction \Lu is empty if and only if we have $q \in \mathbb N$ and  $h \in P ^ q$ such that $ u_{h_1}\ldots u_{h_q}  = v_{h_1}\ldots v_{h_q}$.
   Hence \Lu is empty if and only if our PCP statement have a solution.
  \end{proof}
  Then we have to show that we can recognize \Lu.
  The following lemma help us to decompose it :
  \begin{lm}
    We can recognize the intersection of two recognized language.
  \end{lm}
  \begin{proof}
   We just have to take the two automatons with the union of the initials states.
  \end{proof}
  \begin{rk}
   The number of equivalence relations of the new automaton is the sum of the two previous ones.
  \end{rk}
  Then we will decompose \Lu\ with the two following languages: 
  \begin{df}[\Ld]
    We define the language \Ld $= \left(\bigcup\limits_{i\in \sm} iu_i\right)^* \# \left(\bigcup\limits_{i\in \sm} iv_i\right)^* \# $
  \end{df}
  \begin{df}[\Lt($\Sigma_1,\Sigma_2$)]
    Let $\Sigma_1,\Sigma_2$ be two alphabets, 
    We define the language \Lt($\Sigma_1,\Sigma_2$) $= \{w\#w'\# \in (\Sigma_1 \cupdot \Sigma_2)^* \mid \Pi_{\Sigma_1}(w) = \Pi_{\Sigma_1}(w')\}$.
    Where $\Pi_{\Sigma_1}(z)$ is the projection on $\Sigma_1$ of a word $z$  (ignoring letters of $\Sigma_2$).
  \end{df}
  By construction we have :
  \begin{lm}
    \Lu\ $=$ \Ld\ $\cap$ \Lt($\Gamma,P$) $\cap$ \Lt($P,\Gamma$).
  \end{lm}
  Let us begin to prove that:  
   \begin{lm}
     We can recognize \Ld.
   \end{lm}
   \begin{proof}
     We will use the two following sublemmas:
  \begin{slm}
    \Ld\ is a regular language.
  \end{slm}
  \begin{proof}
    By definition.
  \end{proof}
  \begin{slm}
    We can recognize a regular language
  \end{slm}
  \begin{proof}
   We take the finite state automata and replace transition by (figure),initial state by (figure) and final states by (figure).
   Moreover the initials state of our $NARA$ is the singleton $\{a_0,x_0,(q_0,x_0)\}$ where $q_0$ is the initial state of the finite automata and $(a_0,x_0)$ as been arbitrarily chosen.
   We can check that the language recognized by the $NARA$ is the initial one.   
  \end{proof}
  Then we can recognize \Ld.
  \end{proof}
  Now we set $\Sigma_1,\Sigma_2$ two finite alphabets and we just have to build an $NARA$ recognizing \Lt$(\Sigma_1,\Sigma_2)$.
  Let us begin to build a skeleton of an automaton :
    \inserttikzh{
  \newcommand{\mon}[2]{\node [move] (#1) [#2]{$On \Sigma_1$}}
\newcommand{\egal}[2]{\node [check] (#1) [#2]{$\sim$}}
\newcommand{\fake}[2]{\node [dashed, transition , rectangle ] (#1) [#2]{}}

\node [etou] (first) {$\wedge$};
% \expl {oe1} {below right  of = first}{odd or even }
% \rt {first};
 \mon{fmon}{left of = first }
 edge [post] (first);
% 
 \node (begin) [draw = none, left of = fmon] {}
 edge [post] (fmon);


 
\expl {rmb} {above of = first}{duplicate($k$)}
edge [pre] node[left] {($k:=1$)} (first)
edge [dashed,loop above] node[above]{($k:= 1-k$)} (rmb) ;


\expl {dl} {right  of = rmb}{deal}
edge [pre] node[above] {(k)} (rmb) ;

\expl {suc1} { right  of = dl}{succesor($k$)}
edge [pre] node[above] {(a)} (dl) ;







\expl {nv1} {right of = suc1}{never}
 \rt {suc1};
% \et {et} {right of =nv1} {}
% \rt {nv1};
% 


\expl {fnd} { right  of = nv1}{found$(k)$}
\rt {nv1};


\expl{eg2} {below of = fnd} {same($a$)}
\rt {fnd};

\expl {suc1} {below of = eg2}{succesor($k$)}
\rt {eg2};


\expl {nv2} {left of = suc1}{never}
\rt {suc1};

\tl {l2} {left of = nv2} {\#}
\rt {nv2};


% 
% \expl {sb} {below  of = et}{check consistency}
% \rt {et}
% edge [post] (l2);



\expl{t0}{right of= first}{deal}
edge [pre] node[sloped,above] {($k:=0$)} (first);

\expl{et2}{right of= t0}{succesor(0)}
edge [pre] node[above] {(a)} (t0);

\expl{nv0}{right of= et2}{never}
\rt {et2}
edge [post]  (eg2) ;





  }{Skeleton}
  Arrows with ($k$) or ($a$) represent multiple arrows with a boolean $k$ ($0$ or $1$) or a letter $a$ in $\Sigma_1$.
  Now let us build blocs.
  \newpage
  \inserttikz{
  
\newcommand{\mon}[2]{\node [move] (#1) [#2]{On $\Sigma_1$}}
\newcommand{\egal}[2]{\node [check] (#1) [#2]{$\sim$}}



\node [draw = none] (nfirst) {};
\mv {nmon}{right of = nfirst}
\rt {nfirst};
\ou {nou}{above of = nmon}{}
\rt {nmon};


\tl {nl}{right of =nou}{\#}
\rt {nou};

\et {net}{left of=nou}{}
\rt {nou}
edge [post] (nmon)
;

\tl {nntl}{left of =net }{\notin \Sigma_1}
\rt {net};	

\et {net2}{right of=nou}{}
\rt {nou}
;

\tl {nntl2}{right of =net2 }{\in \Sigma_1}
\rt {net2};	


\expl {ddl}{draw=none,below of = nntl2}{}
\rt{net2};





  }
  {On $\Sigma_1$, move until the first letter in $\Sigma_1$.}
  \inserttikz{
  
\newcommand{\mon}[2]{\node [move] (#1) [#2]{On $\Sigma_1$}}
\newcommand{\egal}[2]{\node [check] (#1) [#2]{$\sim$}}



\node [draw = none] (nfirst) {};
\mon {nmon}{right of = nfirst}
\rt {nfirst};
\ou {nou}{right of = nmon}{}
\rt {nmon};


\tl {nl}{right of =nou}{\#}
\rt {nou};

\et {net}{above of=nou}{}
\rt {nou}
;

\ntl {nntl}{ right of =net }{\#}
\rt {net};	

\expl {ddup}{left of = net}{Duplicate($1-k$)}
\rt {net}
edge [post,bend right,dashed] (nmon)
;


\expl {ddl}{draw=none,below right of = net}{}
edge [pre] node[above] {(k)} (net) ;





  }{Duplicate($k$), open a thread for every letter before the first $\#$.}
  \inserttikz{
  \newcommand{\mon}[2]{\node [move] (#1) [#2]{On $\Sigma_1$}}
\newcommand{\egal}[2]{\node [check] (#1) [#2]{$\sim$}}




\node [draw = none] (dfirst) {};
\st {dst}{right of = dfirst}
\rt {dfirst};

\ou {dou}{right of = dst}{$\Sigma_1$: }
\rt{dst};


\et {det}{right of = dou}{$a$: }
\rt{dou};

\tl {dta}{above right of = det}{$a$}
\rt{det};


\node (dqak) [draw= none, right of = det ]{}
edge [pre] node[above] {(a)} (det) ;



\node (dqak1) [transition,below right of = dou, dashed ]{b}
\rt{dou};


\node (dqakn) [transition,above right of = dou, dashed ]{c}
\rt{dou};

  }{Deal, send the thread at the good branch (according to the current letter). }
  \inserttikz{
  

\newcommand{\mon}[2]{\node [move] (#1) [#2]{On $\Sigma_1$}}
\newcommand{\egal}[2]{\node [check] (#1) [#2]{$\sim$}}
\node [draw = none] (sfirst) {};
\mon {smon}{right of = sfirst}
\rt {sfirst};

\et {set}{right of = smon} {}
\rt {smon};

\eq {seq}{above left of = set }{k}
\rt {set};

\peq {speq}{above right of = set }{1-k}
\rt {set};

\node [draw = none, right of = set] (nlast) {}
\rt{set};



  }
  {Successor($k$), check that the data registered is $\sim_k$  but not $\sim_{1-k}$ to the current one.}  
  \inserttikz{
  
\newcommand{\mon}[2]{\node [move] (#1) [#2]{On $\Sigma_1$}}
\newcommand{\egal}[2]{\node [check] (#1) [#2]{$\sim$}}



\node [draw = none] (nfirst) {};
\mon {nmon}{right of = nfirst}
\rt {nfirst};

\et {net}{right of =nmon } {}
\rt {nmon};


\peq {npeq0}{above right of = net}{0}
\rt {net};


\peq {npeq}{above of = net}{1}
\rt {net};

\ou {nou}{right of = net}{}
\rt {net};


\et {net2}{right of = nou}{}
\rt {nou};

\node [draw = none, right of = net2] (nlast) {}
\rt{net2};


\tl {nl}{below right of =net2}{\#}
\rt {net2};


\et {net3}{below left of = nou}{}
\rt {nou}
edge [post](nmon)
;

\ntl {nntl}{left of =net3 }{\#}
\rt {net3};





  }
  {Never, check that the data registered is not $\sim_0$ and not $\sim_1$ to all data until the $\#$.}
  \inserttikz{
  
\newcommand{\mon}[2]{\node [move] (#1) [#2]{On $\Sigma_1$}}
\newcommand{\egal}[2]{\node [check] (#1) [#2]{$\sim$}}

\node [draw = none] (nfirst) {};
\mon {nmon}{right of = nfirst}
\rt {nfirst};



\ou {nou}{right of = nmon}{}
\rt {nmon};

\et {nlet}{right of =nou } {}
\rt {nou}
;

\eq {neq}{above right of =nlet}{\overline k}
\rt {nlet};


\et {net}{above of=nou}{}
\rt {nou}
edge [post](nmon)
;

\peq {npeq}{left of =net }{\overline k}
\rt {net}
;


\peq {npeq2}{right of =net }{ k}
\rt {net}
edge [pre](nlet)
;



\node [draw = none, right of = nlet] (nlast) {}
\rt{nlet};





  }
  {Found($k$), find the first data $\sim_{1-k}$ to the registered one.}
  \inserttikz{
  

\newcommand{\mon}[2]{\node [move] (#1) [#2]{On $\Sigma_1$}}
\newcommand{\egal}[2]{\node [check] (#1) [#2]{$\sim$}}



\node [draw = none] (sfirst) {};
\mon {smon}{right of = sfirst}
\rt {sfirst};

\et {set}{right of = smon} {}
\rt {smon};

\egal {seg}{above right of = set }
\rt {set};


\tl {sl}{above left of = set }{a}
\rt {set};




\node [draw = none, right of = set] (nlast) {}
\rt{set};


  }
  {Same($a$), check the data is equivalent to the current one and the current letter is $a$ as expected.}    
  Now let us prove that:
  \begin{lm}
    Our automata recognize \Lt$(\Sigma_1,\Sigma_2)$.
  \end{lm}
  \begin{rk}
  Intuitively to recognize $w\#w'\#$ with $ \Pi_{\Sigma_1}(w) =\Pi_{\Sigma_1}(w')$, we will open, for all letter of $\Sigma_1$ in $w$, a thread which check that: 
  \begin{itemize}
   \item this data won't appear again in $w$, 
   \item this data appears exactly one time in $w'$,
   \item when it appears, it is on the same letter of $\Sigma_1$.
  \end{itemize}
  \end{rk}

  \begin{proof}
    Let $z$ be a word recognized by this automaton, $t= |z|$ be the length of $z$.
    For $i \in \stt$ we will write $x(i)$ (respectively $l(i)$) the data (respectively the letter) of the $i^{th}$ letter of $z$ which is in $\Sigma_1 \cupdot \{ \# \}$.  
    We begin to study block by block.
    We will write: $i_b$ the position at the beginning of the block,$i_e$ the position at the end of the block and $x_r$ the data registered at the beginning of the block. 
    Then we can check that we have for $a \in \Sigma_1$ and $k \in \{0,1\}$ :
    \begin{itemize}
      \item [Successor($k$)] 
      $x_r \sim_{k} x(i_b+1)$, \\\
      $x_r \not \sim_{(1-k)} x(i_b+1)$.
      \item [Never] 
      $l(i_e) = \#$ \\\
      $\forall i \in \sbe$,  $x_r \not \sim_{0} x_i$, \\\
      $\forall i \in \sbe$,  $x_r \not \sim_{1} x_i$.
      \item[Found($k$)] 
      $x_r \sim_{(1-k)} x(i_e)$ \\\
      $\forall i \in \sbe - e$,  $x_r \not \sim_{0} x_i$, \\\
     $\forall i \in \sbe - e $,  $x_r \not \sim_{1} x_i$.
     \item[Same($a$)] 
     $x_r \sim_{0} x(i_b+ 1)$, \\\
     $x_r \sim_{1} x(i_b+ 1)$, \\\
     $l(i_e) = a$.
    \end{itemize}
    Moreover we can check that $z=w\#w'\#$ with no occurrence of $\#$ in $w$ and $w'$.
    Then we will write $s=|\Pi_{\Sigma_1}(w)|$, $s'= |\Pi_{\Sigma_1}(w')|$, $S = \ses$ and $S' = \sesp$ .
    With those formulas we can deduce that for all $i \in S$ (respectively $S'$) , for all $j \in S $ (respectively $S'$) 
    if $|i-j| > 1 $ and $k = i\ mod\  2$ we have:
    \begin{itemize}
      \item if $i < s $ (respectively $ < s'$) then $x(i) \sim_k x(i +1)$,
      \item if $i < s $ (respectively $ < s'$) then $x(i) \not \sim_{(1-k)} x(i +1)$,
      \item $x(i) \not \sim_{0} x(j)$ and
      \item $x(i) \not \sim_{1} x(j)$.
    \end{itemize}
    Moreover we can deduce that : for all $i \in S$ if $i > 0$ and $k = i\ mod\  2$ there exist $i' \in S'$ such that:
    \begin{itemize}
      \item for all $j' \in S'$ such that $j'< i' - 1$, $x(i) \not \sim_0 x(j')$ and $x(i) \not \sim_1 x(j')$,
      \item $x(i) \not \sim_k x(i' -1)$ and $x(i) \sim_{(1-k)} x(i' -1)$,
      \item $x(i) \sim_0 x(i')$, $x(i) \sim_1 x(i')$ and $l(i) = l(i')$
      \item and for all $j' \in S'$ such that $j'> i' + 1$, $x(i) \not \sim_0 x(j')$ and $x(i) \not \sim_1 x(j')$.
    \end{itemize}
    Hence for all $i \in S$ if $i > 0$ and $k = i\ mod\  2$ there exist an unique $i' \in S'$ such that $x(i) \sim_0 x(i')$, $x(i) \sim_1 x(i')$ and $l(i) = l(i')$.
    So, as $x(0) \sim_0 x(s+1)$ and  $x(0) \sim_1 x(s+1)$, by recurrence, we can prove $i' = i + s +1$ (with the previous notations).
    Then $s= s'$ and $\Pi_{\Sigma_1}(w) = \Pi_{\Sigma_1}(w')$.
    Hence $z \in$ \Lt$(\Sigma_1,\Sigma_2)$. \\\
    Reciprocally if we take $z \in$ \Lt$(\Sigma_1,\Sigma_2)$, we have $z = w\#w'\#$, with the same notation we can set data as for all $i \in \ses$:
    $x(i) \sim_0 i / 2 \sim_0 x(i+s+1)$ and $x(i) \sim_1 i/2 + (i\ mod \ 2) \sim_1 x(i+s+1)$. We can easily check that, with those data, $z$ is recognized by our automaton. 
    Hence, our automata recognize \Lt$(\Sigma_1,\Sigma_2)$. 
  
  \end{proof}
  Then if we assume that our problem is decidable, we can solve all PCP statements.
  Hence the emptiness problem for \textit{NARA} on $n$-data is undecidable.
  
\end{document}


