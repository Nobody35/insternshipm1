\documentclass{beamer}
\usetheme{Singapore}
\usepackage{default}
\usepackage[utf8]{inputenc}
\usepackage[version=0.96]{pgf}


%tikz
\usepackage{tikz}
\usetikzlibrary{arrows,backgrounds,calc,trees,shapes,snakes,automata,backgrounds,petri}


\newcommand{\cmdtikz}
{
\tikzstyle{transition}=[circle,draw=black!75,minimum size=1cm]
  \tikzstyle{etou}=[transition,circle,fill=blue!75]
  \tikzstyle{check}= [transition,regular polygon,regular polygon sides=3,fill=red!75]
  \tikzstyle{store}= [transition,diamond,fill=green!75]
  \tikzstyle{move}= [transition,rectangle,fill=yellow!75]
  \tikzstyle{expl}= [transition,rectangle]
 }
\newcommand{\st}[2]{\node [store] (#1) [#2]{$Store$}}
\newcommand{\mv}[2]{\node [move] (#1) [#2]{$Move$}}
\newcommand{\expl}[3]{\node [expl] (#1) [#2]{#3}}

\newcommand{\et}[3]{\node [etou] (#1) [#2]{#3 $\wedge$}}
\newcommand{\ou}[3]{ \node [etou] (#1)  [#2] {#3 $\vee$}}

\newcommand{\eq}[3]{\node [check] (#1) [#2]{$eq_{#3}$}}
\newcommand{\peq}[3]{\node [check] (#1) [#2]{ $\overline{eq_{#3}}$}}
\newcommand{\tl}[3]{\node [check] (#1) [#2]{ $#3$}}
\newcommand{\ntl}[3]{\node [check] (#1) [#2]{ $\overline{#3}$}}
\newcommand{\rt}[1]{edge [pre] (#1)}

%insert tikz in beamer
\newcommand{\inserttikz}[2]{
\begin{center}
  \begin{figure}
    \scalebox{0.6}{
    \begin{tikzpicture}[node distance=2cm,>=stealth',bend angle=45,auto]
    \cmdtikz
    #1
\end{tikzpicture}

}

\caption{#2}
\end{figure}
\end{center}
}


%conex hull
\pgfdeclarelayer{background}
\pgfsetlayers{background,main}


\newcommand{\convexpath}[2]{
[   
    create hullnodes/.code={
        \global\edef\namelist{#1}
        \foreach [count=\counter] \nodename in \namelist {
            \global\edef\numberofnodes{\counter}
            \node at (\nodename) [draw=none,name=hullnode\counter] {};
        }
        \node at (hullnode\numberofnodes) [name=hullnode0,draw=none] {};
        \pgfmathtruncatemacro\lastnumber{\numberofnodes+1}
        \node at (hullnode1) [name=hullnode\lastnumber,draw=none] {};
    },
    create hullnodes
]
($(hullnode1)!#2!-90:(hullnode0)$)
\foreach [
    evaluate=\currentnode as \previousnode using \currentnode-1,
    evaluate=\currentnode as \nextnode using \currentnode+1
    ] \currentnode in {1,...,\numberofnodes} {
-- ($(hullnode\currentnode)!#2!-90:(hullnode\previousnode)$)
  let \p1 = ($(hullnode\currentnode)!#2!-90:(hullnode\previousnode) - (hullnode\currentnode)$),
    \n1 = {atan2(\x1,\y1)},
    \p2 = ($(hullnode\currentnode)!#2!90:(hullnode\nextnode) - (hullnode\currentnode)$),
    \n2 = {atan2(\x2,\y2)},
    \n{delta} = {-Mod(\n1-\n2,360)}
  in 
    {arc [start angle=\n1, delta angle=\n{delta}, radius=#2]}
}
-- cycle
}




\begin{document}

\begin{frame}{titre}
\inserttikz{
\node [transition] (q0)  {$q0$};
  \mv {1}{below of= q0} 
    edge [pre] (q0);
  \ou {2}{below of= 1} {$q_{a \vee b}:$ } 
    edge [pre] (1);

    %right%
     
   \et {qb}{right of = 2 } {$q_b$ :}
     \rt  {2};
   \tl{tb}{right of= qb}{b}
     \rt {qb};
   \ou {qki}{below of= qb }{$q_{e}$ :} 
     \rt{qb};
   \eq{eq}{below of= qki}{}
     \rt{qki};
   \mv{move2}{right of= qki}
     \rt{qki}
     edge [post] (qb);
   
   
    %left%
    
   \et {qa}{left of = 2 } {$q_a$:}
      \rt {2};
   \tl{ta}{left of= qa}{a}
     \rt {qa};
   \et{qc}{below of= qa }{$q_{ck}$ :}
     \rt{qa};
   \mv{move}{right of= qc}
     edge [post](2)
     \rt{qc};
     
   \st{store}{below of=qc}
    \rt{qc}
    edge [post] (move);
     \peq{peq}{left of= qc}{}
     \rt{qc};
  
  
 

  \begin{pgfonlayer}{background}
  \pause
  %if
  \expl{ca}{left of= ta}{case a};
  \fill[blue,opacity=0.5] \convexpath{ca,ta,qa,qa.south,qa,2}{23pt};
  \pause
  \expl{cb}{right of= tb}{case b};
  \fill[blue,opacity=0.5] \convexpath{2,qb,qb.south,qb,tb,cb}{23pt};
  
  %create
  \pause
  \expl{cs}{below of= store}{create};
  \fill[green,opacity=0.5] \convexpath{qc,store,qc,cs}{23pt};
  %erase
  \pause
  \expl{ck}{below of= eq}{erase};
  \fill[black,opacity=0.5] \convexpath{qki,eq,ck}{23pt};
   
  %kill
  \pause
  \expl{cc}{left of= peq}{kill};
  \fill[red,opacity=0.5] \convexpath{qc,peq,cc}{23pt};
   
  
  \end{pgfonlayer}
}
{$a^nb^m$}

\end{frame}



\end{document}

