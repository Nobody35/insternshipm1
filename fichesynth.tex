
\documentclass{article}

\usepackage[francais]{babel} %
\usepackage[T1]{fontenc} %
\usepackage[latin1]{inputenc} %
\usepackage{a4wide} %
\usepackage{palatino} %


\newtheorem{td}{TODO}

\let\bfseriesbis=\bfseries \def\bfseries{\sffamily\bfseriesbis}


\newenvironment{point}[1]%
{\subsection*{#1}}%
{}

\setlength{\parskip}{0.3\baselineskip}

\begin{document}

\title{Le titre de votre rapport \\
Pas plus du recto-verso pour cette fiche}

\author{Votre nom, celui de votre encadrant, le nom de son
  �quipe/labo}

\date{La date}

\maketitle

\pagestyle{empty} %
\thispagestyle{empty}

%% Attention: pas plus d'un recto-verso!


\begin{point}{Le contexte g�n�ral}
  Usually, studying automatas and formal languages, we work with on a finite alphabet $\Sigma$.
  But words over infinite alphabets are useful models in a variety of contexts in verification and databases.
  That's why we can wonder what happens when we add to the alphabet $\Sigma$ an infinite set $D$.
  An element of $D$ will be called data, and words on $\Sigma \times D$ will be called data words..
  Precisely I will study automatas which are able to register a data and compare it to the current one.
  The emptiness decidability have been proved for those automatas on those three articles: 
    \begin{td}
     quotes 3 articles
    \end{td}
\end{point}

\begin{point}{Le probl�me �tudi�}
  In my internship I try to extend that result to $n-$data words (words on $\Sigma \times D ^n$). 
\end{point}

\begin{point}{La contribution propos�e}
  I chose to transpose the second proof to my problem. 
  \begin{td}
   quote
  \end{td}

 
\end{point}

\begin{point}{Les arguments en faveur de sa validit�}
  I add to introduce the concept of $n-$nested words to restrict $n-$data words.
  Then I prove two main theorems : 
  \begin{itemize}
   \item The emptiness problem is decidable if we treat $n$-nested words. 
   \item The emptiness problem is undecidable if we treat $n$-data words. 
  \end{itemize}

  
\end{point}


\begin{point}{Le bilan et les perspectives}
  We can note that if we are restricted to $n$-nested words we can find an automata which recognize the language $\{ a^nb^m \ \mid \ n \leq m \}$.
  Compared to that if we can treat every $n$-data words we are able to recognize languages similar to $\{ a^nb^n \}$ or palindromes languages.
  It could be interesting to wonder if there exists an other restriction which allows a better expressiveness without sacrificing the emptiness decidability. 
  To be more general it would be interesting for me to study relations between expressiveness and decidability.
  
\end{point}


\end{document}




