  \documentclass[a4paper,10pt]{report}
\usepackage[utf8]{inputenc}


\usepackage{amsmath}
\usepackage{amssymb}
\usepackage{amsthm}


%barrer hachurer 
\usepackage{ulem}

\newtheorem{thr}{Theorem} %[section]
\newtheorem{lm}{Lemma}
\newtheorem{slm}{Sub-lemma}[lm]
\newtheorem{sslm}{Sub-sub-lemma}[slm]
\newtheorem{pp}{Proposition}
\newtheorem{df}{Definition}

\newtheorem{td}{TODO}

\theoremstyle{remark}
\newtheorem{rk}{Remark}



\newcommand{\dmap}[5]{
#1~:~\begin{array}{ccccc}
#2 &\to& #3 \\
#4  &\mapsto& #5
\end{array}}
\newcommand{\seg}[1]{\{#1\}} 
\newcommand{\segw}[1]{\seg{0..(\mid #1\mid -1)}} 
\newcommand{\ts}{\seg{0 .. (n-1)}}
\newcommand{\C}{\mathcal{C}_{A}}
\newcommand{\I}{\mathcal{I}_{A}}
\newcommand{\F}{\mathcal{F}_{A}}

\newcommand{\U}{\mathcal{U}_{A}}
\newcommand{\Xa}{\mathcal{X}}
\newcommand{\X}{\mathcal{X}^{n}}
\newcommand{\Ya}{\mathcal{Y}}
\newcommand{\Y}{\mathcal{Y}^{n}}
\newcommand{\pb}{$\mathcal{P}_{A}$}

% Title Page
\title{} 
\author{}


%

\setcounter{secnumdepth}{2}


\begin{document}
\maketitle

\section{Introduction and definitions.}

\begin{td}
  Introduction
\end{td}

\subsection{Data words and register automata}

To begin with we define what kind of word and automata we will be studying.

\begin{df}[An $n-$data word, $n-$nested word]
 For $n \in \mathbb{N}$ and $\Sigma$ an alphabet we call $n-$nested word a tuple  
 $\langle \sigma , \sim   \rangle$
 such that:
  \begin{itemize}
   \item $\sigma \in \Sigma^* $ is a word
   \item $(\sim_k)_{k \in \ts }$ is a sequence of equivalence relations on position of $\sigma$. 
  \end{itemize}
  If moreover the sequence of relation is progressively finer which means that:  
  $$\forall k \in \ts, \forall i,j \in \segw{\sigma}, \ (i \sim_k j \Rightarrow \left ( \forall k' \leq k, i \sim_{k'} j \right )) $$ 
   we call the word $n-$nested word.
\end{df}


\begin{df}[Nested Register Automaton \textit{NARA}$_n$]
  A Nested Alternating Register Automaton (\textit{NARA}$_n$) is a tuple : 
  $\langle \Sigma ,Q ,q_0, \delta   \rangle$
  \begin{itemize}
   \item $\Sigma$ is an alphabet,
   \item $Q$ is a finite set of states,
   \item $q_0$ is the initial state ,
   \item $ \delta $ is a map from $Q $ to $\mathcal{T} $, where $\mathcal{T}$ is defined by the grammar : 
   $$ a \ | \  \overline{a} \ | \ Store(q) \ | \ eq_k \ | \  \overline{eq_k} \ | \  q \wedge q' \ | \ q \vee q' \ | \ Move(q)  $$
  where $a \in \Sigma$, $k \in  \ts$ and $q,q' \in Q$.
  \end{itemize}

\end{df}
\begin{rk}
  As we will see: 
  \begin{itemize}
   \item $ a \ | \  \overline{a}$ compare the current letter to $a$.
   \item $eq_k \ | \  \overline{eq_k}$ compare the current data to the registered one (just the $k^{th}$ term).
   \item $Store(q)$ register the current data.
   \item $Move(q)$ move on the word.
   \item $q \vee q' $ allow a choice between two states.
   \item $q \wedge q'$  duplicate a thread in two different one. 
  \end{itemize}
 
\end{rk}
After the definition of configurations we will be able to see examples.


\subsection{The first main theorem.}
  We are now able to state the main theorem : 
\begin{thr}[Emptiness decidability.]
  \label{main}
 The emptiness problem of \textit{NARA} on nested words is decidable.
\end{thr}


\subsection{How our automata works.}

Before proving Theorem~\ref{main}, we have to define things on \textit{NARA} such as data, configurations, transitions, runs...


In the register we will save data defined by :  
\begin{df}[Data]
  For $n\in \mathbb N$, we will need to remember the $n$-tuple of equivalence classes of integers (which will represent a position in a word).
  However we can say that we do not put in the registers the whole class tuple but a tuple of representatives (for example in $\mathbb N$) .
  So we also call data a $n$-tuple of representatives.
  We will write $\Xa$ for the (infinite) set of representatives, and $\X$ the set of data.
\end{df}

\begin{rk}
 Registering just a representative allows us not to guess the length of the word when we are reading it.
\end{rk}

To understand how an automaton works we have to define configurations and runs: 

\begin{df}[Configurations of an \textit{NARA}$_n$ and threads]
For $A$ an \textit{NARA}$_n$ we call configuration on $A$ a tuple  
 $\langle  a ,x_c, \Delta    \rangle$
 such that 
  \begin{itemize}
    \item $a \in \Sigma$ is the current letter, 
    \item $x_c\in  \X$ is the current data, 
    \item $\Delta \in \mathcal{P}_f(\X \times Q)$ is a finite set of threads, where a thread is a pair  
    $\langle q,x  \rangle$ such that:
      \begin{itemize}
	\item $q \in Q$ is the current state of the thread,
	\item $x\in  \X $ is the data in the register of the thread.
      \end{itemize}
   \end{itemize}
 We write $\C =  \Sigma \times \X \times \mathcal{P}_f(\X \times Q)$, the set of all configurations.
\end{df}



\begin{df}[$\rightarrow$, $Move(\Delta)$ and ``$\Delta$ is moving'']
  Let $A = (\Sigma ,Q ,q_0, \delta )$ be an \textit{NARA}$_n$
  Let $\rho = (a,x,\Delta) \in \C$ and $\rho' \in \C$ be two configurations.
  We will define $\rho \rightarrow_A \rho'$ as one of this case: 
   \begin{itemize}
    \item [$  \rho {\xrightarrow{(a',x')} }_M \ \rho'$  (move)] 
    If $\Delta$ is moving (means that for all $(q_1,x_1) \in \Delta$, there exist $q_2 \in Q$ such that $ \delta(q_1) = Move(q_2)$)
    then $\rho'  = (a',x',Move(\Delta))$ 
    where $Move(\Delta) = \{(x_1,q_2) \mid \exists (x_1,q_1) \in \Delta \mid \phi(q_1) = Move(q_2)\}$ and $(a',x') \in (\Sigma \times \X)$.

    \item [$\rho \rightarrow_{\epsilon}  \rho'$ (epsilon)  ] Else we write $\Delta = \{(q_1,x_1)\} \cup \Delta_2$ and we have  $\rho' = (a,x,S \cup \Delta_2 )$ where $S$ depends on $\delta(q_1)$:  
    \begin{itemize}
    \item [$\rho \rightarrow_\vee  \rho'$ (or)] If $\delta(q_1) = (q_2 \vee q_3)$  then we have two possible transitions, so two possible $S$:
      \begin{itemize}
	\item $S = \{ (q_2,x_1)\}$,   
	\item $S = \{(q_3,x_1) \}$.
      \end{itemize}
    \item [$\rho \rightarrow_\wedge  \rho'$ (and)] If $\delta(q_1) = (q_2 \wedge q_3)$  then $S = \{(q_2,x_1); (q_3,x_1) \} $.
    \item [$\rho \rightarrow_s  \rho'$ (store)] If $\delta(q_1) = Store(q_2)$  then $S =  \{(q_2,x_1)\}$.
 
    \item [$\rho \rightarrow_c \rho'$ (check)] Else $S = \emptyset$ if and only if we are in one of those cases:  
	\begin{itemize}
 	 \item $\delta(q_1) = b$ and $ a= b$,
 	 \item $\delta(q_1) = \overline b$ and $ a \neq b$,
 	 \item $\delta(q_1) = eq_j$ and $x_j = {(x_1)}_j$,
 	 \item $\delta(q_1) = \overline{eq_j}$ and $x_j \neq {(x_1)}_j$.
	\end{itemize}
	 (else this thread will be dead, see Remark~\ref{dead})
    \end{itemize}
   

    
   \end{itemize}


\end{df}

\begin{rk}
  With this formalism we do not have the formula $ \phi \wedge \phi$. But we are able to simulate it.
\end{rk}


\begin{rk}
  \label{dead}
  The possibility to have multiple threads forces us to make them moving simultaneously on the word.
  Moreover the acceptance condition will be the lack of thread, so, when $\rightarrow_c$ does not work, the thread will never move.
  We can see such a thread as a ``dead'' thread which avoids the run to be accepted.
\end{rk}
Let us define the end of basis of \textit{NARA}$_n$:
\begin{df}[$\I,\F$, runs, acceptance]
Let $A = (\Sigma ,Q ,q_0, \delta )$ be a \textit{NARA}$_n$. We define :
\begin{itemize}
 \item $\I \subset \C$  is a finite set of configuration called initials.
 \item $\F = \Sigma \times \X \times \{\emptyset\}$ the set of accepting configurations.
\end{itemize}
A run is a finite sequence $\rho \in \C^l$ with $l \in \mathbb N$ and for all $i \in \seg{0..(l-2)}$, $\rho_i \rightarrow \rho_{i+1}$.
Let $l \in \mathbb N$ be an integer and $\rho \in \C^l$  be a run.
We set $m \in \seg{1..(l-1)}$ the number of moving transitions ($\rightarrow_M$), then we have an ordered sequence of indexes $i \in {\mathbb N}^m$ and a sequence of pair $\alpha = {(a_j,x_j)}_{j \in \seg{0..(m-1)}} \in {(\Sigma \times \X)}^m $
such that for all $j \in \seg{0..(m-1)}$, $\rho_{i_j} \xrightarrow{(a_{i_j},x_{i_j})} \rho_{i_j+1}$.
We set $\sigma$ the word obtained by concatenation of the $a_{i_j}$, and $\sim$ the $n$-tuple of equivalence relations induced by the $x_{i_j}$.
Then we write $\rho_0 \xrightarrow{(\sigma,\sim)}$*$ \rho_{l-1}$.
Moreover if $\rho_0 \in \I$ and $\rho_{l-1} \in \F$ we say that $(\sigma,\sim)$ is accepted.



\end{df}


\begin{td}
  Examples
\end{td}

\subsection{An effective transition relation}

To improve the readability we define:
\begin{df}[$d_k(\Delta),Data(\Delta)$]
  Let $A = (\Sigma ,Q ,q_0, \delta )$ be a \textit{NARA}$_n$.
  For $(k,x,\Delta) \in (\ts \times \X \times 2^{Q\times \X} $ we write $d_k(\Delta) = \{r \in \Xa \mid \exists (q,x) \in \Delta \mid r = x_k\}$, 
  and $Data(\Delta) = \bigcup_{k=0}^{n-1} d_k(\Delta)$.
\end{df}

\begin{rk}
 $Data(\Delta)$ is the set of data in $\Delta$ and 
 $d_k(\Delta)$ is the set of $k^{th}$ terms of element of $Data(\Delta)$. 
\end{rk}

\begin{df}[Transition effective]
  If $\rightarrow$ is a transition relation from $C$ to $C$ we will say that $\rightarrow$ is effective when for all $\rho \in C$ we can compute the set 
  $\{ \rho' \in C \mid \rho \rightarrow \rho'\}$
 
\end{df}


\begin{rk}
  We can see that ${\xrightarrow{(a,x)}}_M$ is not effective because $\X$ is infinite.
  In fact we have a finite choice for $x$, because for all $ k \in \ts$ we can chose an $x_k$ in one of the threads or a ``fresh one''.
  Every ``fresh one'' that we can chose are similar (see the relation $\equiv$ that we will define), so we can count it just one time.
  So we will now assume that $\Xa = \mathbb N$ to define an effective version of ${\xrightarrow{}}_M$:
\end{rk}

\begin{df}[New definition of $\rightarrow_M$]
  \label{minipb}
  Let $A = (\Sigma ,Q ,q_0, \delta )$ be a \textit{NARA}$_n$.
  For $\rho = (a,x,\Delta) \in \C$ and $\rho' = (a',x',\Delta') \in \C$.
  We have $\rho {\xrightarrow{(a',x')}}_M \rho'$ if and only if :
  $\Delta$ is moving, $\Delta' = Move(\Delta)$ and $\forall k \in \ts,\ x'_k \in (Data(\Delta) \cup \{max(Data(\Delta))+k\}) $.
  Moreover for all $y \in Data(\Delta) \cup \{x \}$ and all $k \in \ts$  $x'_k = y_k $ implies that for all $l \leq k $,  $x'_l = y_l $.
\end{df}

\begin{rk}
 $max(Data(\Delta))+k$ is just a way to have a ``fresh'' integer.
 Then we could chose another set for $\Xa$ and another way to have ``fresh'' representatives (but here we stay with integers).
 Notice that integer registered are not the position of the word, they are just some representative the ``smallest known''.
 Moreover notice that, with the last restriction, the equivalence relation induced by the word readed is nested. 
\end{rk}



\section{The proof of the main theorem}

\subsection{A way to see the theorem with an order relation}

Now we can remark that:
\begin{lm}
  \label{lmtth}
  Resolving the emptiness problem of an automaton $A$ is equivalent to answer to the problem \pb:
 \begin{itemize}
     \item[Input:] $A$ an \textit{NARA}$_n$.
     \item[Question:] Does there exist $a \in$ $\I$\ and $b \in\F$ such that $ a \rightarrow^* b $?
    \end{itemize} 
\end{lm}
\begin{proof} 
  By definition of the acceptance of a \textit{NARA}$_n$. 
\end{proof}

Then we will continue with some consideration on the transition system $(\C,\rightarrow)$ using the following proposition about relations and transition systems.
For that we need those definitions about transition systems and order relations:

\begin{df}[rdc, $\uparrow_\preceq$ and $ Succ^*_\rightarrow$, $\preceq$ is decidable]
  Let ($C,\rightarrow$) be a transition system and ($C,\preceq$) an order relation.
  We say that ($C,\rightarrow$) is reflexive downward compatible (rdc) with respect of ($C,\preceq$) when :
 for all configurations $\rho,\rho_2,\rho_3 \in C$ such that  $\rho \leq \rho_2$ and $ \rho_2 \rightarrow \rho_3$ 
 there exist a configuration $\rho_1 \in C$ such that  $\rho_1 \leq \rho_3$ and $\rho \rightarrow \rho_1 $ (or $\rho = \rho_1$). 
 \begin{td}
  figure
 \end{td}
  Let $S \subseteq C$,
  we set the two sets of ``successor''  $\uparrow_\preceq S = \{c \in C \mid \exists s \in S \mid s \preceq c \}$, and $Succ^*_\rightarrow S = \{c \in C \mid \exists s \in S \mid s \rightarrow^* c \}$.\\
  To end with we say that $\preceq$ is decidable
   \begin{itemize}
     \item[Input:] $rho,\rho' \in C$ 
     \item[Question:] Do we have $\rho \preceq \rho'$?
    \end{itemize} 
\end{df}




\begin{pp} 
  \label{pp}
  If  ($C,\rightarrow$) is a transition system effective rdc with respect of ($C,\preceq$), a well quasi order decidable, then for any finite set $I\subseteq C$ it's possible to compute a finite set $U \subseteq C$ such that
  $\uparrow_\preceq U=\ \uparrow_\preceq Succ^*_\rightarrow (I)$.
\end{pp}

\begin{td}
  (quote p2.7)
\end{td}

\subsection{Dividing the theorem into several lemmas.}

The following lemma show us what remains to be shown to prove the Theorem~\ref{main}.
\begin{lm}
    \label{mlm}
    If for all \textit{NARA}$_n$ $A$ we can define a relation $\preceq$ such that: 
     \begin{itemize}
	\item ($\C$,$\preceq$) is a wqo,
	\item ($\C$,$\preceq$) is decidable,
	\item ($\C$,$\rightarrow$) is effective,
	\item ($\C$,$\rightarrow$) is rdc with respect to ($\C$,$\preceq$),
	\item $\F$ is downward closed,
	\item $\F$ is recursive (means that we can decide if an element is in $\F$).
     \end{itemize}
      Then the emptiness problem on \textit{NARA} is decidable (Theorem ~\ref{main}).
\end{lm}

\begin{proof} 
  Let $A$ be an \textit{NARA}$_n$,
  $\I$\ is finite and, assuming conditions of the lemma, we can apply Proposition~\ref{pp}.
  Hence me can compute $\U$ finite such that $\uparrow_\preceq \U=\uparrow_\preceq Succ^*_\rightarrow (\I)$.
  Let us begin to show that :
  \begin{slm}	
    \pb' is decidable, where \pb' is the problem : For a given \textit{NARA}$_n$ $A$, does $\U \cap \F =\emptyset$ ? 
  \end{slm}
  \begin{proof}
  $\F$ is recursive and $\U$ is finite so we can test for every element $u \in \U$, if $u \in\F$. Hence we can test if $\U \cap\F=\emptyset$ . 
  In other words the problem \pb' is decidable.
  \end{proof}
  Now let us show that:
  \begin{slm}	
    We can reduce \pb\ to \pb'.
  \end{slm}
  \begin{proof}
  Let $A$ be an \textit{NARA}$_n$.
  If there exist $u \in \U \cap \F$, we have \\\ $u\in \U \subseteq \uparrow_\preceq  \U =\uparrow_\preceq Succ^*_\rightarrow (\I)$. 
  Then by definition of $Succ^*_\rightarrow (\I)$, there exist $a\in$ $\I$\ and $b \in\C$ such that $a \rightarrow^* b \preceq u$.
  Moreover $\F$ is downward closed so, as $u \in \F$,  $b \in \F$.
  \paragraph{}
  Reciprocally let us assume that there exist $a\in$ $\I$\ and $b \in\F$ such that $a \rightarrow^* b $,
  Then $b \in  Succ^*_\rightarrow (\I) \subseteq \uparrow_\preceq Succ^*_\rightarrow (\I)=\uparrow_\preceq  \U$. 
  Then there exist $v \in \U$ such that $v \preceq b$, so, as $\F$ is downward closed, $v \in \F$, then $\U \cap \F \neq \emptyset$.
  \paragraph{}
  Hence the set $\U \cap \F$ is not empty if and only if there exist  $(a,b)\in \I \times \F$ such that $a \rightarrow^* b$.
  In other words, we can reduce \pb\ to \pb'.  
  \end{proof}
  To end with, for all $A$  \textit{NARA}$_n$, we can compute $\U$ and reduce \pb\ to \pb, then, as \pb' is decidable, \pb\ is decidable, hence, according to Lemma~\ref{lmtth}, the emptiness problem on \textit{NARA} is decidable.
\end{proof}

Now let $A =(\Sigma ,Q ,q_0, \delta )$ be an \textit{NARA}$_n$, moreover we introduce some notations to improve the readability.

\begin{df}[$\Ya, \Y, \phi(x),\Delta^k,\Delta^k(r),\Delta(x), \X(\Delta),\phi(x), \phi(\Delta)$ and $ \phi(\rho)$]
We write $\Ya$ for the set of all bijections on $\Xa$  and
for all $(x,\phi) \in (\X \times \Y)$ we define $\phi(x) = (\phi_i(x_i))_{i \in \ts} $. \\\  
Moreover for all $(x,r,k,q,\Delta,\rho,\phi) \in \X \times \Xa \times \ts \times Q \times 2^{Q \times \X} \times \C \times \Y$, we write 
\begin{itemize}
\item $\Delta^k = \{(q,x') \in (Q \times \X) \mid \exists  (q,x') \in \Delta \mid x'_k = x_k\}$,
\item $\Delta^k(r) = \{q \in Q \mid \exists  (q,x) \in \Delta \mid r = x_k\}$,
\item $\Delta(x) = {(\Delta^k(x_k))}_{k \in \ts}$,
\item $\X(\Delta) = \{ x\in X \mid \exists q\in Q \mid (q,x) \in \Delta \}$,
\item $\phi(\Delta) = \{ (q,\phi(x)) \in Q \times \X \mid (q,x) \in \Delta \}$ 
\item and $\phi(\rho) = (a,\phi(x),\phi(\Delta))$ if $\rho = (a,x,\Delta)$.
\end{itemize}
\end{df}


According to Lemma~\ref{mlm} it remains to define $\preceq$ and prove the six conditions (Lemma~\ref{cd4} to Lemma~\ref{cd6}).

\subsection{An order relation consistent with the transition.}
\begin{df}
  Let $(\rho,\rho_2) \in$  $\C\times\C$.
  We write $\rho \equiv \rho_2$ when there exist $\phi \in \Y$ such that  $ \rho = \phi(\rho_2)$.    
  Moreover we write $\rho \leq \rho_2$ when $ \rho=(a ,x, \Delta  )$, $\rho_2=(a ,x, \Delta_2  )$ and $\Delta \subseteq \Delta_2$.
  To finish we write $\rho \preceq \rho_2$ when there exist $\rho' \in \X$ such that $\rho \equiv \rho'$ and $\rho' \leq \rho_2$.
\end{df}



\begin{rk}
  Intuitively we define such an order ''not to disturb`` the transition.
  On the first hand we can see $\equiv$ as a renaming of data' terms. Then it won't change checkings. Moreover it does not involve states so won't change other transitions. 
  On the second hand $\rho \leq \rho'$ mean that $\rho'$ have more thread. So if we have a transition from $\rho'$, either we are able to do the same transition 
  or the tread (ore one of the thread) involved by the transition on $\rho'$ is not in $\rho$. 
  Moreover the acceptance condition is the lack of thread so $\F$ will be trivially downward closed. 
\end{rk}

Formally let us prove that :

\begin{lm} \label{cd4}
  ($\C$,$\rightarrow$) is rdc with respect to ($\C,\preceq$).
\end{lm}


\begin{proof}
  We begin to look $\leq$ with proving that :
  \begin{slm}
  $\forall (\rho,\rho_2,\rho_3) \in C^3  ((\rho \leq \rho_2) \wedge (\rho_2 \rightarrow \rho_3)) \Rightarrow (\exists \rho_1 \in C \mid ((\rho \rightarrow \rho_1 \vee \rho = \rho_1) \wedge \rho_1 \preceq \rho_3))$.
 
  \end{slm}
   \begin{proof}
   Let $\rho = (a,x,\Delta) \in \C$, $\rho_2 = (a_2,x_2,\Delta_2) \in \C$ and $\rho_3 = (a_3,x_3,\Delta_3) \in \C$ be three configurations 
   such that $\rho \leq \rho_2$ and $\rho_2 \rightarrow \rho_3$. We are looking for $\rho_1 \in \C$ such that ($\rho \rightarrow \rho_1$ or $\rho = \rho_1$) and $\rho_1 \leq \rho_3$.
   To begin with, by definition of $\leq$ we have $a = a_2$, $x= x_2$ and $\Delta \subseteq \Delta_2$, then we set $S = \Delta_2 \setminus \Delta$. Let us distinct cases according to $\rightarrow$: 
   \begin{itemize}
    \item [if $\rho_2 \rightarrow_{\epsilon} \rho_3$,] 
    then we can write, as in the definition of $\rightarrow_{\epsilon}$,  $\Delta_2 = \{\alpha \} \cup \Delta'_2$ and $\Delta_3 = S_3 \cup \Delta'_2 $.
    Moreover, by definition of $\rightarrow_{\epsilon}$, $a_3 = a_2$ and $x_3 = x_2$.
    Now, as $\Delta \subseteq \Delta_2$, we can distinct again two cases:

      \begin{itemize}
	\item [if $\alpha \in \Delta$,] then we write $\Delta = \{\alpha \} \cup \Delta'$.  
	We do the same transition on $\rho$ to have $\rho \rightarrow \rho_1$, so we have $\rho_1 = (a,x,\Delta_1)$ where $\Delta_1 = \Delta' \cup S_3$. 
	We have $\Delta' \subseteq \Delta_2$ so $\Delta_1 \subseteq \Delta_3$.
	Moreover $a_3= a_2 = a$ and $x_3 = x_2 = x$.
	Hence $\rho_1 \leq \rho_3$, and we saw that $\rho \rightarrow \rho_1$.
	\item [Else ($\alpha \notin \Delta$)] we set $\rho_1 = \rho$. Then we have $a_3= a_2 = a$ and $x_3 = x_2 = x$, moreover $\alpha \notin \Delta $ so $\Delta \subseteq \Delta_2' \subseteq \Delta_3$.
	Hence $\rho_1 \leq \rho_3$, and $\rho_1 = \rho$ 
      \end{itemize}
     \item [Else ($\rho_2 \xrightarrow{(a_3,x_3)}_{M} \rho_3$)],  
     by definition of $\rightarrow_{M}$, $\Delta_2$ is moving so, as $\Delta \subseteq \Delta_2$, $\Delta$ is moving to, then we can set $\Delta_1 = Move(\Delta)$.
     Now, by definition of $\rightarrow_{M}$, $\Delta_3 = Move(\Delta_2)$ so,
     as $\Delta \subseteq \Delta_2$, $\Delta_1 \subseteq \Delta_3$ .
     Moreover we set for all $k \in \ts$, ${(x_1)}_k = {(x_3)}_k$ if ${(x_3)}_k \in Data(\Delta_1)$ and a fresh representative else.
     Then we can define $f \in \Y$ such that $f(x_1) = x_3$ and invariant elsewhere, then by definition, $f(\Delta_1) = \Delta_1$.
     Hence we just have to set $\rho_1 = (a_3,x_1,\Delta_1)$ to see that $\rho \xrightarrow{(a_3,x_1)} \rho_1$ and $\rho_1 \preceq \rho_3$ (cause $f(\rho_1) \leq \rho_3$).
    \end{itemize}   
  \end{proof}
  Now we look $\equiv$ :
  \begin{slm}
    ($\C$,$\rightarrow$) is rdc with respect to ($\C,\equiv$).
  \end{slm}
  \begin{proof}
    Same as the precedent proof: we set $\rho = (a,x,\Delta) \in \C$, $\rho_2 = (a_2,x_2,\Delta_2) \in \C$ and $\rho_3 = (a_3,x_3,\Delta_3) \in \C$ be three configurations
   such that $\rho \equiv \rho_2$ and $\rho_2 \rightarrow \rho_3$, and we are looking for $\rho_1 \in \C$ such that $\rho \rightarrow \rho_1$  and $\rho_1 \equiv \rho_3$.
   By definition of $\equiv$ we have $f$ bijection on $\X$ such that $\rho_2 = f(\rho)$, so we can set for all $k \in \ts$, $g_k$ = $f_k^{-1}$.
   We can distinct three cases:
    \begin{itemize}
    \item [$\rho_2 \rightarrow_{c} \rho_3$]. By definition g is a tuple of bijection on $\Xa$ so for all $(q',x',k) \in \Delta \times \ts$ we have 
    $x'_k = (x_3)_k$ if and only if  $g(x')_k = {g(x_3)}_k$. Hence, whatever is checked, $\rho \rightarrow g(\rho_3)$.  
    \item [Else if ($\rho_2 \xrightarrow{(a_3,x_3)}_{M} \rho_3$)], as in the precedent proof, we set $\Delta_1 = Move(\Delta)$ and, for all $k \in \ts$, ${(x_1)}_k = {(x_3)}_k$ if ${(x_3)}_k \in \Xa^k(\Delta_1)$ and a fresh representative else.
     Then, as before, we have $h \in \Y$ such that $h(x_1) = x_3$ and $h(\Delta_1) = \Delta_1$.
     Hence we just have to set $\rho_1 = (a_3,x_1,\Delta_1)$ to see that $\rho \xrightarrow{(a_3,x_1)} \rho_1$ and $\rho_1 \equiv \rho_3$ (cause $g(\rho_3) = h(\rho_1)$).
    \item [Else,] the transition is not involved by data's so we can do the same to have $\rho \rightarrow g(\rho_3)$.
    \end{itemize}
    Hence for all those cases, exists $\rho_1 \in \C$ such that $\rho \xrightarrow{} \rho_1$ and $\rho_1 \equiv \rho_3$. 
    In other words ($\C$,$\rightarrow$) is rdc with respect to ($\C,equiv$).
  \end{proof}
  With those two sub-lemmas we can finish the proof as it show in the figure: 
  \begin{td} 
   Picture
  \end{td}
  Formally, let $(\rho,\rho_2,\rho_3) \in C^3 $ such that $ \rho \preceq \rho_2 $ and $ \rho_2 \rightarrow \rho_3$.
  By definition of $\preceq$, there is $\rho' \in \C$ such that $\rho \equiv \rho'$ and $\rho' \leq \rho_2$.
  According to the first sub-lemma, there is $\rho'_1 \in \C$ such that $\rho'_1 \preceq \rho_3$ and ($\rho' \rightarrow \rho'_1$ or $\rho' = \rho'_1$).
  If $\rho' = \rho'_1$ we set $\rho_1 = \rho$.
  Else, according to the second sub-lemma, there is $\rho_1 \in \C$ such that $\rho_1 \equiv \rho'_1$ and ($\rho \rightarrow \rho_1$ or $\rho =\rho_1$).
  Hence there is $\rho_1 \in C $ such that ($\rho \rightarrow \rho_1$ or $\rho = \rho_1$) and $\rho_1 \preceq \rho_3$.
  In other word ($\C$,$\rightarrow$) is rdc with respect to ($\C,\preceq$).

\end{proof}

\subsection{The second main lemma.}
Now we have to prove the second main lemma :

\begin{lm} \label{cd1}
  ($\C$,$\preceq$) is a wqo.
\end{lm}

\begin{rk}
  As ($\C$,$\preceq$) is obviously reflexive and transitive, we will take an infinite sequence of configuration and try to found two comparable elements.
  Intuitively, the difficulty is in the infinity of $\X$, indeed we have an infinity number of different set of threads which are not include each other.
  But, we have to see that $2^Q$ is finite so, even if we have an infinite number of $\Delta$ and of $x$, we have a finite number of projections $\Delta^k(x)$.
  So we will define a map indexed by $k$ and $S$ (a subset of $Q$), which associates to a set of threads the set of component of data $r$ such that $\Delta^k(r) = S$.
  Here we still have ''to much``  partitions of $\X$,  so we won't be able to compare those projections.
  But are able to compare cardinals of those projections.
  This is sufficient because, even if we are looking for $\Delta_i$ included in $\Delta_j$, we can rename data with a tuple of bijection.
  Then we will be able to find two configurations in the sequence such as, $\Delta_i$ is (after a renaming) included in $\Delta_j$.
  By the way, we will use the Dickson's Lemma to see all comparisons of projections as a product of comparisons.
  There is a last difficulty because even if we will find those two set of threads we do not know anything about the letter and the data of the configuration.
  But we can, before doing all that, extract, from our initial sequence, a sequence of configurations $(a,x,\Delta)$ with the same pair $(a,\Delta(x))$.
  With those three ideas let us prove that ($\C$,$\preceq$) is a wqo.
\end{rk}


\begin {proof}
Formally, let $\rho \in \C^{\mathbb N}$ be an infinite sequence of configurations of $A$ and let us show that we could find $(i,j) \in \mathbb{N}^2$ such that $\rho_i \preceq \rho_j$.
To begin with we define
$\dmap{f}{\C}{\Sigma \times {(2^Q)}^n}{(a,x,\Delta)}{(a,\Delta(x))}$.
As $\Sigma \times {(2^Q)}^n$ is finite, by the pigeonhole principle, we can extract from $\rho$ a sequence $\tau$ and find $(a_0,\Delta_0) \in \Sigma \times Q$ such as 
$\forall i \in \mathbb N,\ f(\tau_i)=(a_0,\Delta_0)$.
Now for $\Delta \subseteq Q \times \X $ and $k \in \ts$ we define $\dmap{g^k_\Delta}{2^Q}{2^X}{S}{\{ r\in \Xa | S=\Delta^k(r) \}} .$
Moreover we define $\Delta \preceq^t \Delta_2$ if and only if $\forall (S,k) \in (2^Q \times \ts) ,\ \lvert g^k_\Delta(S) \lvert \leq_{\mathbb N \cup \{\infty\}} \lvert g^k_{\Delta_2}(S) \lvert$. Let us prove that: 
\begin{slm}
  $\preceq^t$ is a wqo.
\end{slm}
\begin{proof}
We will see $\preceq^t$ as a product, for that we set $ \dmap{h}{2^{Q\times X}}{({\mathbb N \cup \{\infty\})}^{(2^Q \times \ts)}}{\Delta}{(|g^k_{\Delta}(S)|)_{(S,k)\in (2^Q \times \ts)  }}.$
Then, by construction, $\Delta \preceq^t \Delta_2$ if and only if $ h(\Delta) \leq_{{(\mathbb N \cup \{\infty\})}^{(2^Q \times \ts)}} h(\Delta_2)$.
Moreover $Q$ is finite so $(2^Q \times \ts)$ is and, as $\leq_{\mathbb N \cup \{\infty\}}$ is a wqo, by Dickson's Lemma $\leq_{{(\mathbb N \cup \{\infty\})}^{(2^Q \times \ts)}}$
is a wqo so $\preceq^t$ is.
\begin{td}
  Quote Dickson's lemma
\end{td}
\end{proof}
Then we can find $(i,j)\in \mathbb{N}^2$ such that $\tau_i=(a_0,x_i,\Delta_i), \tau_j=(a_0,x_j,\Delta_j)$ and $\Delta_i \preceq^t \Delta_j$.
Now let us prove that :

\begin{slm}
  There exist $\tau_i' \in \C$ such that $\tau_i \equiv \tau_i'$ and $\tau_i' \leq \tau_j$.
\end{slm}
\begin{proof}
We saw that $\forall (S,k) \in (2^Q \times \ts),\ \lvert g^k_{\Delta_i(S)} \lvert \leq_{\mathbb N \cup \{\infty\}} \lvert g^k_{\Delta_j}(S) \lvert$, 
so for all $\forall (S,k) \in (2^Q \times \ts)$ we can define an injection $\phi^k_S: g^k_{\Delta_i}(S) \to g^k_{\Delta_j}(S)$. 
Moreover $\Delta_i(x_i)=\Delta_j(x_j)=\Delta_0$ then we can, defining $\phi_{\Delta_0}$, set that $\phi_{\Delta_0}(x_i) = x_j$.
Now we define, for all $k \in \ts$, a map $\dmap{\psi_k}{d_k(\Delta_i)}{\psi_k(d_k(\Delta_i))}{r}{\phi^k_{\Delta_i(r)}(r)}$, let us show that:
\begin{sslm}
  For all $k \in \ts$, $\psi_k$ is bijective.
\end{sslm}
\begin{proof}
Let $k \in \ts$.
Let $(r,r') \in \Xa^2 $ such as $\psi(r)= \psi(r') = y$.
By definition of $\psi$ there is $(S,S')\in {2^Q}^2$ such that  $y = \phi_S^k(x) =\phi^k_{S'}(x')$. So $y\in g^k_{\Delta_j}(S)$ and  $y\in g^k_{\Delta_j}(S')$.
Then, by definition of $g$,
$S=\Delta_j(y)=S'$, moreover $\phi^k_S$ is injective we have $ r=r'$.
Hence  $\psi_k$ is injective.
Moreover $\psi_k$ is injective and defined from $Data(\Delta_i)$ to $\psi_k(Data(\Delta_i))$.
In other words: $\psi_k$ is bijective.
\end{proof}
Now we set $\Delta_i'=\psi (\Delta_i)$ and $\tau_i'=\psi (\tau_i)$. 
Moreover written $ \tau_i'=(a_0,x_i',\Delta'_i)$ we can see that, by construction, $ x_i'=\psi(x_i)=x_j$, hence $\tau_i \equiv \tau_i'$.
It's remains to show that:
\begin{sslm}
  $\tau_i' \leq \tau_j$.
\end{sslm}

\begin{proof}
  We have just seen that $ x_i'=\psi(x_i)=x_j$, so we just have to show that $\Delta_i' \subseteq \Delta_j$,
  then we set $(q,x) \in \Delta_i'$, and we try to show that $(q,x) \in \Delta_j$. 
  Let us begin to set $k \in \ts$. 
  By construction of $\Delta_i'$, there is $r \in d^k(\Delta_i)$ such as $(q,r) \in \Delta^k_i$ and $x_k = \psi^k(r)$.
  By definition of $\psi^k$, $x_k=\phi_{\Delta_i^k(r)}(r) \in g_{\Delta^k_j}(\Delta^k_i(r))$ 
  then $ x_k \in \{ s \in \Xa | \Delta^k_j(s)= \Delta^k_i(r) \}$.
  So $\Delta^k_i(r)=\Delta^k_j(x_k)$, especially, as $q \in \Delta^k_i(r)$, we have  $q \in \Delta^k_j(x_k)$.
  Hence for all $k \in \ts$ there exist $z \in \X$ such that $(q,z) \in \Delta_j$ and $z_k = x_k$.
  Especially for $k = n-1$, we take those $z \in \X$.
  As we treat with $n$-nested word, $z_{(n-1)} = x_{(n-1)}$ imply $z = x$. 
  Then $(q,x) \in \Delta_j$.
  Hence for all $(q,x') \in \Delta_i'$ we have $(q,x') \in \Delta_j$, in other word $\Delta_i' \subseteq \Delta_j$ and $\tau_i' \leq \tau_j$.
\end{proof}
Then we have $\tau_i \equiv \tau_i'$ and  $\tau_i' \leq \tau_j$.
\end{proof}	
Hence $\tau_i \preceq \tau_j$, then for all $\rho \in \C^{\mathbb N}$ infinite sequence of configurations we can find $i < j$ such that $\rho_i \preceq \rho_j$.
Moreover $\preceq$ is reflexive and transitive, by construction, then $\preceq$ is a wqo.

\end{proof}


\subsection{The last lemmas.}
\begin{lm} \label{cd2}
  ($\C$,$\preceq$) is decidable.
\end{lm}
\begin{proof}
  Sets of threads are finite, so they contains a finite number of data, then there is a finite number of tuple of bijections to test before testing an inclusion between two finite sets.
  Hence $\preceq$ is decidable.
\end{proof}

\begin{lm} \label{cd3}
  ($\C$,$\rightarrow$) is effective.
\end{lm}

\begin{proof}
   Sets of threads are finite, and for all case there is a finite number of transition possible (definition ~\ref{minipb}).
   Hence ($\C$,$\rightarrow$) is effective.
\end{proof}


\begin{lm} \label{cd5}
  $\F$ is downward closed. 
\end{lm}
\begin{proof}
  Let $\rho=(a,x,\Delta) \in \C$ a configuration such that there exist $\rho' \in \F$ with $\rho \preceq \rho'$.
  By definition of $\preceq$ and $\F$, there exist $f$ in $\Y$ such that $f(\Delta) \subseteq \emptyset$.
  Then $\Delta$ is empty, in other words $\rho \in \F$.
  Hence $\F$ is downward closed. 
\end{proof}

\begin{lm} \label{cd6}
  $\F$ is recursive.
\end{lm}

\begin{proof}
  It's just testing if $\Delta$ is empty.
\end{proof}

\begin{td}
 Conclude and introduce second theorem.
\end{td}


\end{document} 
  