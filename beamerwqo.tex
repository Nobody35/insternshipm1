\documentclass{beamer}
\usepackage{tikz}
\usetikzlibrary{matrix}
\newtheorem{df}{Definition}
\newtheorem{thr}{Theorem} %[section]
\newtheorem{lm}{Lemma}[thr]
\begin{document}

\begin{frame}{$\preceq$ is a well quasi order.}
 \begin{itemize}
  \pause
  \item Reflexive and transitive \pause : take $\rho$ and look for $\rho_i \preceq \rho_j$ \pause
  \item Data $\infty$, Threads $\infty$ \pause : use $g^k_{\Delta}(S)$.  \pause
  \item Still infinite \pause : use cardinals. \pause
  \item $\forall S, \forall k $: use Dickson's lemma. \pause
  \item $\Delta_i$ and $\Delta_j$ \pause $\exists f $ $f(\Delta_i) \subseteq \Delta_j$ \pause
  \item But what about $(a,x)$? \pause : project on $(a,\Delta(x))$. \pause
  \item Found $\rho_i$ and $\rho_j$ \pause $\exists f $ $f(\rho_i) \preceq \rho_j$. \pause 
  \item $\preceq$ is a well quasi order.
 \end{itemize}

\end{frame}

\begin{frame}{Last lemmas.}
\begin{lm} 
  $\preceq$ is decidable.
\end{lm}
\pause
\begin{lm} 
  $\rightarrow$ is effective.
\end{lm}
\pause
\begin{lm} 
  $F$ is downward closed. 
\end{lm}
\pause
\begin{lm} 
  $F$ is reccursive.
\end{lm}

\end{frame}

% 
\end{document}